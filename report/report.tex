\documentclass{article}



\usepackage[latin1]{inputenc}
\usepackage{color}
\usepackage{minted} 


\begin{document}
    \title{Proyecto de Probabilidades} 
	\author{Daniel de la Cruz Prieto}
	\date{\today}    
    \maketitle
    
    \section*{Ejercicio 2 }  
    {\bf Una persona conocer\'a pretendientes hasta encontrar su pareja ideal y
    casarse. Si al conocer a un pretendiente, la probabilidad de que sea su pareja ideal
    es p. Simule la cantidad de pretendientes que debe conocer hasta casarse.}
    
    \paragraph{Respuesta} 

    Aqui voy a usar una distribucion binamial geom\'etrica  pues si defino la v.a.d como 
    el numero de ensayos necesarios hasta que se obtiene el primer \'exito . Ademas de 
    que todos los eventos suceden con probabilidad $p$ y la probabilidad en cada una de 
    los ensayos es la misma 

    \begin{equation}
        P \left(X = x \right) = p \left(1-p\right)^{k-1} 
    \end{equation}


    \begin{minted}[linenos = true , frame = lines , framesep  = 5mm , label = Generar la distribucion Binomial Geometrica ]{Python} 
    import matplotlib.pyplot as plt 
    import numpy as np 
    from scipy import stats 
    #import seaborn as sns 

    #Graficando geometrica 
    p = 0.3 # paramtro de forma 

    geometrica  = stats.geom(p); 

    x = np.arange(geometrica.ppf(0.01) , geometrica.ppf(0.99))
    print(x) 
    fmp = geometrica.pmf(x); #Funcion de masa de probabilidad 

    plt.plot(x , fmp , '--'); 
    plt.vlines(x,0,fmp , colors ='b' , lw = 5 , alpha = 0.3) 

    plt.title ( 'Distribucion Geometrica con parametro  p = 0.5 '  )
    plt.ylabel ('P (X = x)') 
    plt.xlabel ('(X = x)')
    plt.savefig('gafica1b.png')
    plt.show() 
    \end{minted}
    
            

	
\end{document}