\documentclass{article}

\usepackage{color}
\usepackage{minted} 
\usepackage{amsmath}

\pagestyle{plain} 

\begin{document}
    \title{Proyecto de Probabilidades} 
	\author{Daniel de la Cruz Prieto}
	\date{\today}    
    \maketitle
    
    \section*{Ejercicio 1 Inciso b }  
    {\bf Una persona conocer\'a pretendientes hasta encontrar su pareja ideal y
    casarse. Si al conocer a un pretendiente, la probabilidad de que sea su pareja ideal
    es p. Simule la cantidad de pretendientes que debe conocer hasta casarse.}
    
    \paragraph{Respuesta} 

    Aqui voy a usar una distribucion binamial geom\'etrica  pues si defino la v.a.d como 
    el n\'umero de ensayos necesarios hasta que se obtiene el primer \'exito . Ademas de 
    que todos los eventos suceden con probabilidad $p$ y la probabilidad en cada una de 
    los ensayos es la misma 

    \begin{equation}
        P \left(X = x \right) = p \left(1-p\right)^{k-1} 
    \end{equation}


    \begin{minted}[linenos = true , frame = lines , framesep  = 5mm , label = Generar la distribucion Binomial Geometrica ]{Python} 
    import matplotlib.pyplot as plt 
    import numpy as np 
    from scipy import stats 
    #import seaborn as sns 

    #Graficando geometrica 
    p = 0.3 # paramtro de forma 

    geometrica  = stats.geom(p); 

    x = np.arange(geometrica.ppf(0.01) , geometrica.ppf(0.99))
    print(x) 
    fmp = geometrica.pmf(x); #Funcion de masa de probabilidad 

    plt.plot(x , fmp , '--'); 
    plt.vlines(x,0,fmp , colors ='b' , lw = 5 , alpha = 0.3) 

    plt.title ( 'Distribucion Geometrica con parametro  p = 0.5 '  )
    plt.ylabel ('P (X = x)') 
    plt.xlabel ('(X = x)')
    plt.savefig('gafica1b.png')
    plt.show() 
    \end{minted}
    
            




    \section*{Ejercicio 3 Clase Pr\'actica 7}

    El tiempo (en horas) que se requiere para que se requiere para repara una m\'aquina es una v.a 
    que sigue una distribucion exponencial con parametro $ \lambda = \frac{1}{2} $ Calcule: 
    \begin{enumerate}
        \item La probabilidad de que el tiempo de reparacion exceda las 2 horas 
        \item La probabilidad de que la reparacion dure al menos 10 horas dado que se conoce que
         la duracion excede las 9 horas 
    \end{enumerate} 
    
    {\bf Respuesta 1 } 
    Como el problema sigue una distribucion exponencial entonces su funci\'on de desidad es : 

    
    \begin{equation*}
        f(x)=\begin{cases}
            \lambda e^{-\lambda x }  & \mbox{si $x>0$,}
            \\
            0                        & \mbox{si $x\le 0$.}
            \end{cases}
    \end{equation*}

    La probabilidad de que el tiempo de duracion exceda las 2 horas se 
    puede obtener primero calculando la probabilidad de que el tiempo de 
    duracion no exceda las 2 horas  y eso esta dado por $P \left(X \le x\right)$. 
    Como la integral es el area bajo la y la funcion $f\left(x\right) $ es positiva 
    entonces : 
    


    \begin{equation*}
        \begin{matrix}
            P \left(X \le 2 \right) & = & \int_{ - \infty   }^{2}  f\left(t\right)\,\mathrm{d}t
            \\\\
                                    & = & \int_{0 }^{2}  \lambda e^{- \lambda t}\,\mathrm{d}t
            \\\\
                                    & = & \int_{0 }^{2}  \frac{1}{2} e^{- \frac{t}{2} }\,\mathrm{d}t
            \\\\
                                    & = & \frac{1}{2} \left[2 \left(1-\frac{1}{e}\right)\right]
            \\\\
                                    & = & 1 - \frac{1}{e}
        \end{matrix}
    \end{equation*}
    

    Luego la probabilidad de que el tiempo de duracion exceda las 2 horas es : 

    \begin{equation*}
        P \left(X > 2 \right) = 1 - \left(1 - \frac{1}{e}\right)  = \frac{1}{e} 
    \end{equation*}


    {\bf Respuesta 2 } 

    Aqui hay que hallar la probabilidad de que suceda un evento $A$ dado que sucede otro $B$.
    Esto es probabilidad condicional definimos los sucesos de la siguiente forma :

    \begin{itemize}
        \item que la duracion exceda las 9 horas (Suceso A )
        \item que la duracion exceda las 10 horas (Suceso B)
    \end{itemize}
    
    Para calcular hallar la probabilidad de $B$ dado $A$ usamos Bayer
    
    \begin{equation*}
        P \left(B | A  \right) = \frac{P \left(AB\right)}{P\left(A\right)}
    \end{equation*}

    Para calcular el suceso $A$ 

    \begin{equation*}
        P\left(A \right) = P \left(x > 9 \right) = 1 - P \left(X \le 9 \right)
    \end{equation*}

    \begin{equation*}
        \begin{array}{rcl}
            P \left(x \le 9 \right) &  =  & \int_{0}^{9} f\left(t\right) \,\mathrm{d}t 
            \\
            \\
                                    &  =  & \int_{0}^{9} \lambda e^{- \lambda t}  \,\mathrm{d}t 
            \\
            \\
                                    &  =  & - e ^{-\lambda t } |_{0}^{9} 
            \\
            \\
                                    & = & -e^{- \lambda \left(9\right)} - \left[e^{-  \lambda \left(0\right) }\right]
            \\
            \\
                                    & = & -e^{-9 \lambda} + 1   
            \\
            \\
                                    & = & 1 - e^{-9 \lambda}   
            \\
            \\
                                    & = & 1 - e^{\frac{-9}{2}}  \mbox{ sustituyendo  $ \lambda = \frac{1}{2}$}
        \end{array}
    \end{equation*}

    Luego podemos obtener la probabilidad de que la duracion exceda las 9 horas 
    
    \begin{equation*}
        P \left(X > 9 \right) = 1 -  P \left(X \le 9 \right)  = 1 - 1 + e^{\frac{-9}{2}} = e^{\frac{-9}{2}}
    \end{equation*}

    Ahora calculamos el suceso $B$ . Aqui hacemos lo mismo que para el suceso $A$ y obtenemos que 

    \begin{equation*}
        P \left(X > 10 \right) = 1 -  P \left(X \le 10 \right)  = 1 - 1 + e^{-5} = e^{-5}
    \end{equation*}

    Ahora tenemos que la probabilidad de $P \left(AB\right) = P \left(B\right) = e^{-5}$
    y podemos calcular la probabilidad de $B$ dado $A$ 

    \begin{equation*}
        P \left(B|A\right) = \frac{e^{-5}}{e^{-\frac{9}{2}}} = e^{-5 + \frac{9}{2}}  = e^{-\frac{1}{2}}
    \end{equation*}

    La probabilidad de que la reparaci\'on dure al menos 10 horas dado que se conoce
    que la duraci\'on excede las 9 horas es $e^{-\frac{1}{2}}$

    \section*{Ejercicio 3 Clase Practica 8}

    Sea  $x$ un variable aleatoria continua y no negativa .Demuetre que 

    \begin{equation*}
        EX = \int_{0}^{+ \infty} P \left(X > x \right)  \,\mathrm{d}x 
    \end{equation*}

    Si $Y$ es una variable aleatoria continua con $fy$ como funcion de densidad .
    Se tiene : 

    \begin{equation*}
        \int_{0}^{\infty} P \left(Y > y\right)  \,\mathrm{d}y =  
        \int_{0}^{\infty} \int_{y}^{\infty}  fy \left(x\right)\,\mathrm{d}x  \,\mathrm{d}y
    \end{equation*}

    Pero tenemos que  

    \begin{equation*}
        P \left(Y > y \right) = 
        \int_{y}^{\infty} fy \left(x\right) \,\mathrm{d}x 
    \end{equation*}

    Intercambiando el orden de integracion de la funci\'on  anterior tenemos que 

    \begin{equation*}
        \begin{array}{rcl}
            \int_{0}^{\infty} P \left(Y > y \right) \,\mathrm{d}y &  =  &  \int_{0}^{\infty} \left(\int_{0}^{x}  \,\mathrm{d}y \right) fy\left(x\right) \,\mathrm{d}x 
            \\
            \\
                                                                  &  =  &  \int_{0}^{\infty} x fy\left(x\right) \,\mathrm{d}x
            \\
            \\
                                                                  &  =  &  E\left[Y\right]                                           
        \end{array}
    \end{equation*}

    \paragraph{Nota:} La demostraci\'on de este ejercicio esta en el "Ross A First Course in Probability $8^{th}$ edition"  p\'agina 195 capitulo 5
\end{document}