\documentclass{article}

\usepackage{color}
\usepackage{minted} 
\usepackage{amsmath}
\usepackage{amssymb} 

\pagestyle{plain} 

\begin{document}
    \title{Proyecto de Probabilidades} 
	\author{Daniel de la Cruz Prieto}
	\date{\today}    
    \maketitle
    
    \section*{Ejercicio 1 inciso a}

    \begin{flushleft}
        {\bf Se realizan n llamadas telef\'onicas independientes en un mismo d\'ia, simule
        el tiempo total de duraci\'on de las llamadas.}

        Para la simulacion de este caso vamos a utilizar una variable aleatoria continua con 
        distribucion Gamma $X  $ que se utiliza para simular una variable aleatoria continua con 
        distribuci\'on exponencial , que se utiliza para simular los tiempos de vida de un 
        sistema determinado , en este caso en particular es la duracion de cada una de las llamadas. 

        \begin{center}
            \begin{tabular}{|l|c|r|}
                \hline
                \multicolumn{3}{|c|}{Resultados de la simulaci\'on}\\
                \hline
                \hline
                $X \thicksim \varGamma \left(\lambda , \alpha\right) $ & $EX$ & promedio \\
                \hline
                $X \thicksim \varGamma \left(\frac{1}{2} , 30\right) $ & 60& $60.1532373091$  \\
                \hline 
                $X \thicksim \varGamma \left(\frac{1}{4} , 20\right) $ & 80& $80.148902903$  \\
                \hline
                $X \thicksim \varGamma \left(\frac{1}{6} , 10\right) $ & 60& $59.7998983886$ \\
                \hline
                $X \thicksim \varGamma \left(\frac{1}{8} , 36\right) $ & 288 & $290.06879581$  \\
                \hline
            \end{tabular}
        \end{center}

        Los resultados de la tabla anterior se obtuvieron de codigo de python siguiente : 

        \begin{minted}{Python}
            import numpy as np 
            import random as rdm

            def exponencial (l):
                U = rdm.random()   
                return -1 / l * np.log(U) 

            def gammasimulation (l , n ): # lambda , numero de llamadas
                total = 0 
                for i in range (n):
                    total += exponencial(l)
                return total     


            def media (l,n,s): #lambda , #llamads  , simulaciones 
                promedio = 0 ;
                for i in range (s):
                    promedio += gammasimulation(l,n)
                return promedio / s     
        \end{minted}
    \end{flushleft}
   
   

    \section*{Ejercicio 1 inciso b }  
    \begin{flushleft}
        
        
        {\bf Una persona conocer\'a pretendientes hasta encontrar su pareja ideal y
        casarse. Si al conocer a un pretendiente, la probabilidad de que sea su pareja ideal
        es p. Simule la cantidad de pretendientes que debe conocer hasta casarse.}
        
        \paragraph{Respuesta}
        Aqui voy a usar una distribucion binamial geom\'etrica  pues si defino la v.a.d como 
        el n\'umero de ensayos necesarios hasta que se obtiene el primer \'exito . Ademas de 
        que todos los eventos suceden con probabilidad $p$ y la probabilidad en cada una de 
        los ensayos es la misma 


        \begin{center}
            \begin{tabular}{|l|c|r|}
                \hline
                \multicolumn{3}{|c|}{Resultados de la simulaci\'on}\\
                \hline
                \hline
                $X  \thicksim  Geo \left(\lambda\right) $ & $EX$ & promedio \\
                \hline
                $X  \thicksim  Geo \left(\frac{1}{2}\right) $ & 2& $2.013$  \\
                \hline 
                $X  \thicksim  Geo \left(\frac{1}{3}\right) $ & 3 & $3.021$  \\
                \hline
                $X  \thicksim  Geo \left(\frac{1}{4}\right) $ & 4 & $3.98$ \\
                \hline
                $X  \thicksim  Geo \left(\frac{1}{5}\right) $ & 5 & $4.965$  \\
                \hline
            \end{tabular}
        \end{center}

        Los valore de la tabla anterior fueron obtenidos con el codigo en python siguiente 

        \begin{minted}{Python} 
            import numpy as np 
            import random as rdm 
            
            def Bernoulli (p):
                U = rdm.random() 
                if U <= p :
                    return 1 
                else:
                    return 0
            
            def GeometricSimulation (p):
                count = 1
                while Bernoulli(p) != 1 :
                    count = count +1 
                return count 
            
            def promedio (p , n ):
                promedio = 0 
                for i in range(n):
                    promedio += GeometricSimulation(p) 
                return promedio /n    
        \end{minted}
        
            
    \end{flushleft}



    \section*{Ejercicio 1 inciso c }
    
    \begin{flushleft}
        {\bf Asuma que la cantidad media de mensajes que env\'ia una persona en un
        dı\'ia tiene forma de campana de Gauss. Esto significa que la probabilidad de que envı\'ie
        pocos o muchos mensajes es baja. Simule la cantidad media de mensajes enviados. }


        \paragraph{Respuesta }En este caso vamos a usar una variable aleatoria con distribucion Normal .
        Como lo que se quiere es conocer la cantidad media de mensajes en un d\'ia y Esto se representa con 
        Campana de Gauss , que se utiliza para encontrar la media 

        
        Para la simulacion halle el promedio de 1000 simulaciones con $\mu = 1 $ $ \sigma = 3$ para un 
        valor esperado de 1 y obtuve un promedio de $1.04345015085$

        Los datos los obtuve utilizando el codigo en python de abajo 


        \begin{minted}{Python}
            import random as rdm
            import numpy as np  
            import math


            def exponencial (l):
                U = rdm.random()   
                return -1 / l * np.log(U) 

            def simulacionNormal (miu , sigma ):
                U = rdm.random()
                Y = exponencial(1) 

            
                while U > math.pow(math.e , (-1)*(math.pow(Y-1,2)/2)):
                    U = rdm.random() 
                    Y = exponencial(1) 

                U2 = rdm.random(); 

                if U2 > 0.5:
                    Y = -Y

                return (Y * sigma) + miu    
        \end{minted}
    \end{flushleft}


    \section*{Ejercicio 3 Clase Pr\'actica 7}

    \begin{flushleft}

        El tiempo (en horas) que se requiere para que se requiere para repara una m\'aquina es una v.a 
        que sigue una distribucion exponencial con parametro $ \lambda = \frac{1}{2} $ Calcule: 
        \begin{enumerate}
            \item La probabilidad de que el tiempo de reparacion exceda las 2 horas 
            \item La probabilidad de que la reparacion dure al menos 10 horas dado que se conoce que
            la duracion excede las 9 horas 
        \end{enumerate} 
        
        {\bf Respuesta 1 } 
        Como el problema sigue una distribucion exponencial entonces su funci\'on de desidad es : 

        
        \begin{equation*}
            f(x)=\begin{cases}
                \lambda e^{-\lambda x }  & \mbox{si $x>0$,}
                \\
                0                        & \mbox{si $x\le 0$.}
                \end{cases}
        \end{equation*}

        La probabilidad de que el tiempo de duracion exceda las 2 horas se 
        puede obtener primero calculando la probabilidad de que el tiempo de 
        duracion no exceda las 2 horas  y eso esta dado por $P \left(X \le x\right)$. 
        Como la integral es el area bajo la y la funcion $f\left(x\right) $ es positiva 
        entonces : 
        


        \begin{equation*}
            \begin{array}{rcl}
                P \left(X \le 2 \right) & = & \int_{ - \infty   }^{2}  f\left(t\right)\,\mathrm{d}t
                \\\\
                                        & = & \int_{0 }^{2}  \lambda e^{- \lambda t}\,\mathrm{d}t
                \\\\
                                        & = & \int_{0 }^{2}  \frac{1}{2} e^{- \frac{t}{2} }\,\mathrm{d}t
                \\\\
                                        & = & \frac{1}{2} \left[2 \left(1-\frac{1}{e}\right)\right]
                \\\\
                                        & = & 1 - \frac{1}{e}
            \end{array}
        \end{equation*}
        

        Luego la probabilidad de que el tiempo de duracion exceda las 2 horas es : 

        \begin{equation*}
            P \left(X > 2 \right) = 1 - \left(1 - \frac{1}{e}\right)  = \frac{1}{e} 
        \end{equation*}


        {\bf Respuesta 2 } 

        Aqui hay que hallar la probabilidad de que suceda un evento $A$ dado que sucede otro $B$.
        Esto es probabilidad condicional definimos los sucesos de la siguiente forma :

        \begin{itemize}
            \item que la duracion exceda las 9 horas (Suceso A )
            \item que la duracion exceda las 10 horas (Suceso B)
        \end{itemize}
        
        Para calcular hallar la probabilidad de $B$ dado $A$ usamos probabilidad condicional 
        
        \begin{equation*}
            P \left(B | A  \right) = \frac{P \left(AB\right)}{P\left(A\right)}
        \end{equation*}

        Para calcular el suceso $A$ 

        \begin{equation*}
            P\left(A \right) = P \left(x > 9 \right) = 1 - P \left(X \le 9 \right)
        \end{equation*}

        \begin{equation*}
            \begin{array}{rcl}
                P \left(x \le 9 \right) &  =  & \int_{0}^{9} f\left(t\right) \,\mathrm{d}t 
                \\
                \\
                                        &  =  & \int_{0}^{9} \lambda e^{- \lambda t}  \,\mathrm{d}t 
                \\
                \\
                                        &  =  & - e ^{-\lambda t } |_{0}^{9} 
                \\
                \\
                                        & = & -e^{- \lambda \left(9\right)} - \left[e^{-  \lambda \left(0\right) }\right]
                \\
                \\
                                        & = & -e^{-9 \lambda} + 1   
                \\
                \\
                                        & = & 1 - e^{-9 \lambda}   
                \\
                \\
                                        & = & 1 - e^{\frac{-9}{2}}  \mbox{ sustituyendo  $ \lambda = \frac{1}{2}$}
            \end{array}
        \end{equation*}

        Luego podemos obtener la probabilidad de que la duracion exceda las 9 horas 
        
        \begin{equation*}
            P \left(X > 9 \right) = 1 -  P \left(X \le 9 \right)  = 1 - 1 + e^{\frac{-9}{2}} = e^{\frac{-9}{2}}
        \end{equation*}

        Ahora calculamos el suceso $B$ . Aqui hacemos lo mismo que para el suceso $A$ y obtenemos que 

        \begin{equation*}
            P \left(X > 10 \right) = 1 -  P \left(X \le 10 \right)  = 1 - 1 + e^{-5} = e^{-5}
        \end{equation*}

        Ahora tenemos que la probabilidad de $P \left(AB\right) = P \left(B\right) = e^{-5}$
        y podemos calcular la probabilidad de $B$ dado $A$ 

        \begin{equation*}
            P \left(B|A\right) = \frac{e^{-5}}{e^{-\frac{9}{2}}} = e^{-5 + \frac{9}{2}}  = e^{-\frac{1}{2}}
        \end{equation*}

        La probabilidad de que la reparaci\'on dure al menos 10 horas dado que se conoce
        que la duraci\'on excede las 9 horas es $e^{-\frac{1}{2}}$

    \end{flushleft}
    
    \section*{Ejercicio 3 Clase Pr\'actica 8}

    \begin{flushleft}
        
    
        Sea  $x$ un variable aleatoria continua y no negativa .Demuetre que 

        \begin{equation*}
            EX = \int_{0}^{+ \infty} P \left(X > x \right)  \,\mathrm{d}x 
        \end{equation*}

        Si $Y$ es una variable aleatoria continua con $fy$ como funcion de densidad .
        Se tiene : 

        \begin{equation*}
            \int_{0}^{\infty} P \left(Y > y\right)  \,\mathrm{d}y =  
            \int_{0}^{\infty} \int_{y}^{\infty}  fy \left(x\right)\,\mathrm{d}x  \,\mathrm{d}y
        \end{equation*}

        Pero tenemos que  

        \begin{equation*}
            P \left(Y > y \right) = 
            \int_{y}^{\infty} fy \left(x\right) \,\mathrm{d}x 
        \end{equation*}

        Intercambiando el orden de integracion de la funci\'on  anterior tenemos que 

        \begin{equation*}
            \begin{array}{rcl}
                \int_{0}^{\infty} P \left(Y > y \right) \,\mathrm{d}y &  =  &  \int_{0}^{\infty} \left(\int_{0}^{x}  \,\mathrm{d}y \right) fy\left(x\right) \,\mathrm{d}x 
                \\
                \\
                                                                    &  =  &  \int_{0}^{\infty} x fy\left(x\right) \,\mathrm{d}x
                \\
                \\
                                                                    &  =  &  E\left[Y\right]                                           
            \end{array}
        \end{equation*}

        \paragraph{Nota:} La demostraci\'on de este ejercicio esta en el "Ross A First Course in Probability $8^{th}$ edition"  p\'agina 195 capitulo 5

    \end{flushleft}

    \section*{Ejercicio 9 Clase Pr\'actica 7} 

    \begin{flushleft}
            
        Sea $ X   \thicksim Exp \left(\lambda\right) $  Pruebe que : 

        \begin{equation*}
            P \left(X-t > x \vert  X > t\right) = P \left(X>x\right),  \hspace*{1.0cm}  x , t \ge 0 
        \end{equation*}

        Tenemos que:
        \begin{equation*}
            P \left(X-t > x \vert  X > t\right) =  P \left(X > x + t\vert  X > t\right)
        \end{equation*}
        
        Ahora por probabilidad condicional tenemos que:

        \begin{equation*}
            P \left(X-t > x \vert  X > t\right) =  \frac{P \left(\left(X>x +t \right) \cup \left(X > t \right)  \right)}{ P \left(X>t\right)}
        \end{equation*}

        Por operaciones de conjunto y teniendo en cuenta que $t \ge 0 $ podemos plantear que:

        \begin{equation*}
            \left(x , \infty \right) \cap \left(x+t , \infty \right)  = \left(x+t , \infty \right)
        \end{equation*} 

        Ahora nos queda que :

        \begin{equation*}
            P \left(X-t > x \vert  X > t\right) =  \frac{P \left(X > x +t \right)}{ P \left(X>t\right)}
        \end{equation*}

        Pero podemos plantear lo siguiente : 

        \begin{equation*}
            \begin{array}{rcl}
                P \left(X > x +t \right)  &  =  &  1 - P \left( X \le x+t \right)
                \\
                P \left(X > t \right)     &  =  &  1 - P \left( X \le t \right)
            \end{array} 
        \end{equation*}

        Luego como sabemos que la funcion de distribucion de la exponencial esta dada por :
        
        \begin{equation*}
            F(x)=\begin{cases}
                1-  e^{-\lambda x }  & \mbox{si $x>0$,}
                \\
                0                    & \mbox{si $x\le 0$.}
                \end{cases}
        \end{equation*}

        Pero como $x > 0  $ entonces $F\left(x\right) = 1 - e^{- \lambda x}$ . si sustituimos y tenemos 
        en cuenta lo anterior nos queda de la siguiente forma la igualdad :

        \begin{equation*}
            \begin{array}{rcl}
                P \left(X-t > x \vert  X > t\right)  & = & \frac{1 - P \left(X \le x + t \right)}{1 - P\left(X \le t \right)}
                \\
                \\
                                                    & = & \frac{1 - F \left( x + t \right)}{1 - F\left(t \right)}
                \\
                \\
                                                    & = & \frac{1 - \left[1 - e^{- \lambda \left(x+t\right)}\right]}{1 - \left[1 - e^{- \lambda \left(t\right)}\right]}
                \\
                \\
                                                    & = & \frac{e^{- \lambda \left(x+t\right)}}{e^{- \lambda\left(t\right)}}
                \\
                \\
                                                    & = & e^{- \lambda x}
                \\
                \\
                                                    & = & 1 - \left[1 - e^{- \lambda x}\right]
                \\
                \\                                   
                                                    & = & 1 - F \left(x\right);
                \\
                \\
                                                    & = & P \left( X > x \right)
            \end{array}        
        \end{equation*}

        Por lo que tenemos que si $ x,t \ge 0 $: 

        \begin{equation*}
            P \left(X-t > x \vert  X > t\right) = P \left( X > x \right) 
        \end{equation*}

        
            
    \end{flushleft}




    \section*{Ejercicio 5 Clase Pr\'actica 8} 


    \begin{flushleft}
        {\bf Un equipo que mide y almacena continuamente la actividad sı\'ismica es colocado
        en una regi\'on remota. El tiempo T , de rotura del equipo sigue una distribuci\'on
        exponencial de media 3 años. Dado que el equipo no se revisa en los primeros
        dos años, el tiempo de detección de rotura es $X = \max (T, 2)$. Determine EX.
        Sugerencia: Use la f\'ormula del ejercicio 3. }

        \paragraph{Respuesta}
        Nos dicen que $ET = 3$ pero como el tiempo $T$ de rotura del equipo sigue una
        distribuci\'on exponencial sabemos que $ET = \frac{1}{\lambda}$  por lo que podemos 
        obtener tambien que $\lambda = \frac{1}{3}$
        
        Del ejercicio 3 de la Clase Pr\'actica 8 tenemos $EX = \int_{0}^{\infty} P\left(X > x\right) \,\mathrm{d}x $ . Usando esto tenemos : 

        \begin{equation*}
            \begin{array}{rcl}
                EX  & = &   \int_{0}^{\infty} P \left(X > x \right) \,\mathrm{d}x 
                \\\\
                    & = &  \int_{0}^{2} P \left(X > x \right) \,\mathrm{d}x + \int_{2}^{\infty} P \left(X >x \right) \,\mathrm{d}x   
            \end{array}
        \end{equation*}

        dado que el tiempo de detecci\'on de rotura esta dado por $X = \max\left(T,2\right)$ y en $\left(0,2\right)$ se cumple siempre que $X>x$
        por lo que $P\left(X>x\right) = 1 $ 

        \begin{equation*}
            \begin{array}{rcl}
                EX  & = &   \int_{0}^{2} P \left(X > x \right) \,\mathrm{d}x + \int_{2}^{\infty} P \left(X >x \right) \,\mathrm{d}x   
                \\\\
                    & = &  \int_{0}^{2}  \,\mathrm{d}x + \int_{2}^{\infty} P \left(X >x \right) \,\mathrm{d}x   
                \\\\
                    & = &  2 + \int_{2}^{\infty} P \left(X >x \right) \,\mathrm{d}x   
            \end{array}
        \end{equation*}

        Podemos comprobar que $X$ no toma valor 2 y adem\'as en el intervalo $\left(2,\infty \right)$ las probabilidades son las mismas 
        $P \left(X>x \right)  = P \left(T>x \right)$ 

        \begin{equation*}
            \begin{array}{rcl}
                EX  & = &   2  + \int_{2}^{\infty} P \left(X >x \right) \,\mathrm{d}x   
                \\\\
                    & = &   2 +  \int_{2}^{\infty} \left(1- F \left(x\right)\right) \,\mathrm{d}x 
                \\\\
                    & = &   2 +  \int_{2}^{\infty} \left(1- \left(1 - e^{- \lambda x }\right)\right) \,\mathrm{d}x   
                \\\\
                    & = &   2 +  \int_{2}^{\infty} \left(1- \left(1 - e^{- \frac{1}{3} x }\right)\right) \,\mathrm{d}x   \hspace*{1.0cm}  \mbox{sustituci\'on de $\lambda = \frac{1}{3}$ }
                \\\\ 
                    & = &   2 +  \int_{2}^{\infty} e^{- \frac{1}{3} x } \,\mathrm{d}x 
                \\\\
                    & = &   2 + 3 e^{-\frac{2}{3}}
            \end{array}
        \end{equation*}

    \end{flushleft}

    \section*{Ejercicio 8 inciso a Clase Pr\'actica 8}

    \paragraph{Respuesta}Algoritmo para desarrollar la Variable Aleatoria 

    \begin{minted}{Python}
        import numpy as np 
        import math
        import random as rdm

        def funcionInversa (y , l , a):
            return (math.log (1-y) / -l)**(1/a)

        def SimulaciondelafuncionInversa (l , a ):
            Y = rdm.random()
            return funcionInversa(Y,l,a)
    \end{minted}

    \section*{Ejercicio 8 inciso b Clase Pr\'actica 8}

    \paragraph{Respuesta}Algoritmo para desarrollar la Variable Aleatoria 

    \begin{minted}{Python}
        import math
        import random as rdm

        def funcionInversa (y):
            if y < 1/2:
                return 1/2 * math.log(2*y)
            else :
                return -0.5 * math.log(-2*y+2) 


        def SimulaciondelafuncionInversa ():
            return funcionInversa(rdm.random())
    \end{minted}

\end{document}